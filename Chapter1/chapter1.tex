\chapter{Related Work}
\label{Sec:RelatedWork}
Determining the winning set of nodes and the winning strategies in a
graph game had been a primary concern in the past, and significant
work has been done for safety, reachability and more general
$\omega$-regular objectives in this
area~\cite{AutomataLogicsInfiniteGames}. Work on cooperation and
rational behavior of players is now an emergent area of
research~\cite{AssumptionSynthesis,PerimeterOfWorldModel,CompositionalSynthesis,AdmissibleInfiniteGames,RationalSynthesis}.

Faella~\cite{AdmissibleInfiniteGames} discussed the potential benefits
of real world applications in playing a game from a \textit{losing}
node in a rational way to achieve the best possible results. His
primary focus was on \textit{best-effort strategies} by relying on the
cooperation of the other player. He outlined a procedure to compute
admissible strategies for \emph{positional} and
\textit{prefix-independent} goals, where the latter are goals closed
under adding or deleting finite prefixes to/from its elements. But he
did not provide a method for goals which are not prefix-independent,
such as safety and reachability. He claimed that if all \textit{Player
	2} nodes are treated as \textit{Player~1} nodes, then the winning
strategies obtained, called \emph{cooperatively winning}, are also
admissible for safety and reachability objectives. It is shown via
examples that this claim is incorrect and algorithms are provided for
finding admissible strategies for safety and reachability objectives,
arguably the most important among all goals which are
non-prefix-independent.

Bloem \etal~\cite{AssumptionSynthesis} discussed various approaches
for dealing with assumptions on environment behavior while solving
games. They proposed four goals which should be met by all system
designs, out of which what is relevant here is the goal of ``not
giving up'' when the system guarantee cannot be enforced under
worst-case assumptions. As mentioned above, in many situations, the
environment may not be perfectly adversarial and provide the
worst-case input. In such cases, it makes sense for the system to try
to meet its goal as much as possible, assuming some level of
cooperation from the environment. The work addresses precisely this
situation, finding admissible strategies for the important cases of
safety and reachability objectives. Note that, by definition,
admissible strategies make the minimal assumptions about cooperation
from the other player.

Damm and
Finkbeiner~\cite{PerimeterOfWorldModel,CompositionalSynthesis}
proposed algorithms for synthesizing and verifying dominant strategies
which is based on construction of tree automata for objectives
specified in LTL.  The notion of dominant strategy considered in the
work is identical to what is called an admissible strategy in this
work. The work is a special case of
\cite{PerimeterOfWorldModel,CompositionalSynthesis}, which considers
safety and reachability objectives. The algorithms developed are also much
simpler, based primarily on graph-theoretic notions and of lower
complexity than the double exponential algorithm in
\cite{PerimeterOfWorldModel,CompositionalSynthesis}.


In \cite{RationalSynthesis}, Fisman \etal\ introduce the idea of
rational synthesis, \ie, synthesis in the context of multiple
autonomous agents where each agent has a goal of its own and the goals
are not necessarily adversarial. The approach provides a strategy not
just for the agent in question (called the \emph{system agent} in
\cite{RationalSynthesis}) but for all the agents such that the
specification of the system agent is satisfied and the strategy
profile for all the agents meet some desired solution concept, such as
the existence of a dominant strategy, Nash equilibrium, or
subgame-perfect equilibrium. In such equilibria, no agent has the
incentive to unilaterally deviate from its strategy. While admissible
strategies are clearly related to the problem of rational synthesis,
exploring the exact relationship is part of future work.
