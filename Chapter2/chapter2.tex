\chapter{Preliminaries}
\label{Sec:Preliminaries}
A \textit{game structure} \textit{G} is defined as a pair $(V, E)$ where
\begin{enumerate}
	\item $V = V_{1} \uplus V_{2}$ represents the finite set of vertices/nodes in the game graph, where $V_{1}$ and $V_{2}$ are the sets of \textit{Player~1} and \textit{Player~2} nodes respectively.
	\item $E\subseteq V \times V$ represents the finite set of edges in the game graph. Every node in V has a successor, \ie, $\forall v \in V, \exists v' \in V . (v,v') \in E$.
\end{enumerate}

A \textit{play} in $G$ is defined as a sequence of nodes $v_{0} v_{1}
v_{2} \ldots$ such that $(v_{i},v_{i+1}) \in E$. If the current node
$v \in V_{1}$, then \textit{Player~1} makes the next move by selecting
an edge $(v,v') \in E$ that takes the game to $v'$. If the current
node $v \in V_{2}$, a \textit{Player~2} move is defined analogously. A
\textit{game} $\mathcal{G} = (G,\mathit{Win})$ is a pair of a game structure
and a winning condition $\mathit{Win}$, given by a set of plays in $G$.

A game can have various winning conditions (like
\textit{B$\ddot{u}chi$}, \textit{Muller}, \textit{Parity}, etc.) that
specify the set $\mathit{\mathit{Win}}$ of plays that are won by
\textit{Player~1}. In the work, only safety and
reachability objectives are considered. For a game structure $G = (V,E)$, a
\textit{safety objective} is given by a set $S \subseteq V$ of
nodes. A play $v_{0}v_{1}v_{2}\ldots$ in $V^{\omega}$ is safe,
\ie, winning for \textit{Player~1}, if it never leaves the set $S$,
\ie, $v_{i} \in S$ for all i$\in \mathbb{N}$. \textit{Player~1} loses
the game if the play enters V$\backslash$S at any point.  A
\textit{reachability objective} is specified by a set $R \subseteq V$
of nodes. The goal for \textit{Player~1} is to force the game to reach
a node in $R$, \ie, a play $v_{0}v_{1}v_{2}\ldots$ is winning
for \textit{Player~1} if $v_{i} \in R$ for some i$\in \mathbb{N}$.

A \textit{strategy} for \textit{Player~1} is defined by a function $
\sigma : V^{*}V_{1} \rightarrow V $ that maps the history of observed
nodes to the next node. A \textit{Player~2} strategy is defined
analogously. Given strategies $\sigma$ and $\rho$ for \textit{Player
	1} and \textit{Player~2} respectively, the work defines \textit{Out}($
\sigma, \rho, v$) as the unique word $v_{0} v_{1} v_{2}..$ in
$V^{\omega}$ such that $v_{i+1} = \sigma(v_{i})$ if $v_{i} \in V_{1}$
and $v_{i+1} = \rho(v_{i})$ otherwise, for all $i \in \mathbb{N}$ and
$v_{0} = v$. The work defines \textit{Out}($\sigma,v$) = \{\textit{Out}($
\sigma, \rho, v) \: | \: \rho$ is a \textit{Player~2} strategy\} as
the set of outcomes from node $v$ when \textit{Player~1} plays
according to strategy $\sigma$. A strategy $\sigma$ for \textit{Player
	1} from node $v$ is \textit{winning} if \textit{Out}($\sigma,v)
\subseteq \mathit{\mathit{Win}}$, where $\mathit{\mathit{Win}}$ is a set of winning
plays, \ie, $\sigma$ is a winning strategy for \textit{Player~1} if
for all \textit{Player~2} strategies $\rho$, \textit{Out}($ \sigma,
\rho, v) \in$ $\mathit{\mathit{Win}}$. The work defines $val(\sigma,\rho,v)$ as 1 if
$Out(\sigma, \rho, v) \in \mathit{\mathit{Win}}$ and 0 otherwise.  For a game
$\mathcal{G}$, a strategy for $\sigma$ for \textit{Player~1} is said
to be \textit{positional} (or \textit{memoryless}) if $\sigma$ depends
only on the last node of the current history of the game \ie, there
is a function $f\: : V_{1} \rightarrow V $ such that
$\sigma(sv_{1})\:= \:f(v_{1})$ for all $s \in V^{*}$. For the winning
conditions of safety and reachability, positional strategies suffice
and hence attention is restricted to such strategies in the rest of the thesis.

The following definitions appear in \cite{OmegaRegularGames}. For any two \textit{Player~1} strategies $\sigma_{1}$ and $\sigma_{2}$, $\sigma_{1}$ \textit{very weakly dominates} $\sigma_{2}$ at node $v$, if $val(\sigma_{1}, \rho, v) \geq val(\sigma_{2}, \rho, v)$.
A \textit{Player~1} strategy $\sigma_{1}$ is \textit{dominated} by another \textit{Player~1} strategy $\sigma_{2}$ at node $v$, if $val(\sigma_{1}, \rho, v) \leq val( \sigma_{2}, \rho, v)$, for all \textit{Player~2} strategies $\rho$ and $val(\sigma_{1}, \rho, v) < val(\sigma_{2}, \rho, v)$ for at least one \textit{Player~2} strategy $\rho$. It is also said $\sigma_{2}$ \textit{dominates} $\sigma_{1}$ at $v$ in such a case.
An \textit{admissible strategy} for \textit{Player~1} is a strategy $\sigma_{1}$ that is not dominated by any other \textit{Player~1} strategy $\sigma_{2}$ at any vertex $v$. 
\begin{figure}[thpb]
	\centering
	\begin{tikzpicture}
	\draw (0,0) ellipse (3cm and 1.5cm);
	\node at (1.3,-0.75) [safeReach] {\large $S$};
	\node at (0,-0.5) [player2] (1) {$v_{2}$};
	\node at (1.25,0.5) [player2] (2) {$v_{1}$};
	\node at (-1.25,0.5) [player1] (3) {$v_{0}$};
	\node at (3,1.75) [player2] (4) {$v_{3}$};
	\draw (4) edge[loop above] (4);
	\draw (3) edge (2);
	\draw (3) edge (1);
	\draw (1) edge (2);
	\draw (2) edge[bend right] (3);
	\draw (2) edge (4);
	\draw (1) edge[bend right] (4);
	\end{tikzpicture}
	\caption{Admissible strategy for safety}
	\label{figurelabel7}
\end{figure}
\begin{figure}[thpb]
	\centering
	\begin{tikzpicture}
	\draw (3,1) ellipse (1.5cm and 1cm);
	\node at (3.5,0.75) [safeReach] {\large $R$};
	\node at (0,-0.25) [player2] (1) {$v_{2}$};
	\node at (0,1.5) [player2] (2) {$v_{1}$};
	\node at (-1.75,0.5) [player1] (3) {$v_{0}$};
	\node at (1.5,-1) [player1] (4) {$v_{3}$};
	\node at (2.5,0.75) [player1] (5) {$v_{4}$};
	\draw (4) edge[loop above] (4);
	\draw (3) edge (2);
	\draw (3) edge (1);
	\draw (2) edge (1);
	\draw (1) edge (4);
	\draw (1) edge (5);
	\draw (2) edge (5);
	\end{tikzpicture}
	\caption{Admissible strategy for reachability}
	\label{figurelabel8}
\end{figure}

Figures~\ref{figurelabel7} and \ref{figurelabel8} depict examples of
admissible strategies for safety and reachability, when there does not
exist a winning strategy for \textit{Player~1} from node
$v_{0}$. Throughout the thesis \textit{Player~1} nodes are depicted by
circles and \textit{Player~2} nodes by squares.
Figure~\ref{figurelabel7} is an example of a game with a safety
objective, where $S = \{v_{0},v_{1},v_{2}\}$. Here \textit{Player~1}
does not have a winning strategy from $v_{0}$ as \textit{Player~2} can
always take the game to $v_{3}$. But the \textit{Player~1} strategy
$\sigma_{1}$, which chooses the edge ($v_{0}$,$v_{1}$) is better than
the strategy $\sigma_{2}$, which chooses the edge ($v_{0}$,$v_{2}$),
because the former performs strictly better than the latter against
the \textit{Player~2} strategy \{$v_{1} \mapsto v_{0}, v_{2} \mapsto
v_{3}, v_{3} \mapsto v_{3}$\}. Similarly for the reachability game
depicted in Figure~\ref{figurelabel8} with $R=\{v_4\}$, there does not
exist a winning strategy for \textit{Player~1} from $v_{0}$. But the
\textit{Player~1} strategy $\sigma_{1}$ which chooses the edge
$(v_{0},v_{1})$ is better than the strategy $\sigma_{2}$ which chooses
the edge $(v_{0},v_{2})$, because the former performs strictly better
than the latter against the \textit{Player~2} strategy \{$v_{1}
\mapsto v_{4}, v_{2} \mapsto v_{3}$\}. These two examples also show
that treating all \textit{Player~2} nodes as \textit{Player~1} nodes
and finding winning strategies for \textit{Player~1} does not identify
the admissible strategies for safety and reachability games as claimed
by Faella~\cite{AdmissibleInfiniteGames}.