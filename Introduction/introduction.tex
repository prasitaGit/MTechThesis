%%% Thesis Introduction --------------------------------------------------
\chapter{Introduction}
\label{Sec:Introduction}
Games played on finite graphs, typically involving two players, have
been a subject of widespread investigation in the field of computer
science, with applications primarily in the field of controller
synthesis~\cite{SynthesisOfReactiveModule}. Typically, a winning
strategy for \textit{Player~1} (the system) against \textit{Player~2}
(the environment) gives rise to a controller that satisfies a given
specification.

The games considered have a set of winning and losing nodes for
\textit{Player~1}. In this work, the primary focus is on losing nodes
and attempt to achieve \textit{best-effort strategies}, also known as
\textit{admissible strategies}~\cite{OmegaRegularGames}. In many
applications finding such a best-effort strategy when a winning
strategy does not exist from a given node is very useful. Instances of
such applications are problems which do not involve a strong sense of
competition, \ie, \textit{Player~2} can be cooperative rather than
adversarial or more realistically when \textit{Player~2} is rational,
\ie, focused more on achieving its own objective rather than thwarting
\textit{Player~1} from meeting its goal. This happens, for instance,
in component-based systems where the components are autonomous agents
such as robots which have their own objectives to meet and are not
necessarily adversarial. In such cases, \textit{Player~1} can end up
winning if this does not prevent \textit{Player~2} from meeting its
goal, even if there is no strategy that will win against all
\textit{Player~2} strategies.

\textit{Safety} and \textit{reachability} are two very important goals
in system design. Safety requires plays to remain forever within a
safe set of nodes, whereas reachability requires a set of nodes to be
visited after finitely many steps. The work computes admissible
strategies for these two types of objectives.

Admissible strategies are also known as \textit{non-dominated
	strategies} \ie, the game can be lost by such strategies if and only
if no other strategy can win the game starting from the same node
against any strategy of \textit{Player~2}. In this work, the
perspective of \textit{Player~1} is taken into consideration. The semantics are defined
analogously when played from \textit{Player~2's} perspective and the
algorithms hold good. The notion of \textit{admissibility} is slightly
different from Faella~\cite{AdmissibleInfiniteGames} and is consistent
with the definition of Brenguier \etal~\cite{OmegaRegularGames}.

The rest of the thesis is organized as follows. First the
related work done in this area is discussed in Chapter~\ref{Sec:RelatedWork},
followed by the preliminaries and definitions
in \ref{Sec:Preliminaries}. Then the two algorithms
developed for safety and reachability objectives are discussed in
Chapters~\ref{Sec:Safety} and \ref{Sec:Reachability} respectively,
along with their time complexity.